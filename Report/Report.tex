\documentclass{article}

% Layout
\usepackage[a4paper,margin=2.5cm]{geometry}
\parindent=0pt
\frenchspacing

% Packages
\usepackage[none]{hyphenat}
\usepackage[english]{babel}
\usepackage{parskip}
\usepackage{hyperref}

% Configuration
\hypersetup{colorlinks,linkcolor=black,citecolor=black,filecolor=black,urlcolor=black}

% Document contents
\begin{document}

\title{Distributed sentiment analysis on GitHub commit comments}
\author{Leon Helwerda (s1034375) and Tim van der Meij (s1115731)}
\date{\today}
\maketitle

\begin{abstract}
    In this report we discuss performing sentiment analysis on GitHub
    commit comments. We describe the problems and solutions related to
    downloading and preprocessing the dataset and gaining insights
    into the data using a naive implementation that uses lists with
    positive and negative words. After manually labeling 2000 training
    examples, we benchmark several classification algorithms and we use
    the most promising classifier in our final implementation where we
    apply the trained classifier on the entire dataset. We visualize
    the overall sentiment per programming language to show the potential
    of our approach. Distributing work is key when working with such a
    large dataset, hence getting the data, preprocessing it and
    performing the sentiment analysis is all done in a distributed
    manner with help of OpenMPI and MapReduce.
\end{abstract}

\section{Introduction}\label{sec:introduction}
\ldots

\section{Problem statement}\label{sec:problem}
\ldots

\section{Dataset}\label{sec:dataset}
\ldots

\section{Implementation}\label{sec:implementation}
\ldots

\section{Experiments}\label{sec:experiments}
\ldots

\subsection{Results}\label{sec:results}
\ldots

\section{Conclusion}\label{sec:conclusion}
\ldots

\end{document}
